\begin{exercise}{43}
Show that for any events $A$ and $B$,

$$P(A) + P(B) -1 \leq P(A \cap B) \leq P(A \cup B) \leq P(A) + P(B)$$

For each of these three inequalities, give a simple criterion for when the inequality is
actually an equality (e.g., give a simple condition such that $P(A \cap B) = P (A \cup B)$ if
and only if the condition holds).

\end{exercise}

\begin{proof}
Will approach this problem algebraically. We know that $P(A \cup B) = P(A) + P(B) - P(A \cap B)$ and hence $P(A) + P(B) = P(A\cup B) + P(A \cap B)$. Replacing this in the leftmost inequality, we obtain

$$ P(A\cup B) + P(A \cap B) -1 \leq P(A \cap B) $$

Subtracting $P(A \cap B)$ from both sides,

$$ P(A\cup B) -1 \leq 0 $$

, which is equivalent to 

$$P(A \cup B) \leq 1$$

, which holds from the basic properties of the probability.

The equality $P(A) + P(B) -1 = P(A \cap B)$ holds if and only if $A \cup B = \Sigma \cup X$, for some $X$ such that $P(X) = 0$. This is because in this case we have $P(A \cup B) = P(A) + P(B) - P(A \cap B) = 1$ and the result follows with a simple rearrangment of the terms.

\vspace{1em}

From set theory, we know that $A \cap B \subseteq A$ and $A \cap B \subseteq B$. Hence, $A \cap B \subseteq A \cup B$ and hence $P(A \cap B) \leq P(A \cup B)$.

In the equality case, we have that $P(A \cup B) = P(A \cap B)$. But $P(A \cup B) = P(A) + P(B) - P(A \cap B)$. From these two equations we obtain that $P(A) + P(B) = 2P(A \cap B)$ which happens if and only if $A\setminus B$ and $B\setminus A$ are both of probability $0$. . 

\vspace{1em}

As it has been discussed in the text, in the relation $P(A\cup B) = P(A) + P(B) - P(A \cap B)$ we subtract $P(A \cap B)$ to account for the double counting of the overlap between $A$ and $B$. It follows directly that $P(A  \cup B) \leq P(A) + P(B)$, with equality when $P(A \cap B) = 0$, which happens if and only if $A \cap B$ is a set of probability 0, such as the empty set.

\end{proof}

\newpage

