\begin{exercise}{25}
A hash table is a commonly used data structure in computer science, allowing for fast
information retrieval. For example, suppose we want to store some people’s phone numbers. Assume that no two of the people have the same name. For each name $x$, a hash
function $h$ is used, letting $h(x)$ be the location that will be used to store $x$’s phone
number. After such a table has been computed, to look up $x$’s phone number one just
recomputes $h(x)$ and then looks up what is stored in that location.


The hash function $h$ is deterministic, since we don’t want to get different results every
time we compute $h(x)$. But $h$ is often chosen to be pseudorandom. For this problem,
assume that true randomness is used. Let there be $k$ people, with each person’s phone
number stored in a random location (with equal probabilities for each location, independently of where the other people’s numbers are stored), represented by an integer
between $1$ and $n$. Find the probability that at least one location has more than one
phone number stored there.
\end{exercise}

\begin{proof}
    The probability of a repeated hash, known as hash collision, represents the probability that a random sample of $k$ values from a range of $n$ values produces at least a duplicated item. This is an instance of the birthday problem, where we were sampling birthday dates from a range of 365 possible values.

    Therefore, the probability of not having a hash collision is $p = \frac{\prod_{i=0}^{k-1} (n-i)}{n^k}$, and therefore the probability of a hash collision is $1-p = 1 - \frac{\prod_{i=0}^{k-1} (n-i)}{n^k}$. This property can be exploited to break cryptographic protocols using the so-called \href{https://en.wikipedia.org/wiki/Birthday_attack}{birthday attack}.
\end{proof}

