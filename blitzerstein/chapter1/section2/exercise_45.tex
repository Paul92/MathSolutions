\begin{exercise}{45}
Let $A$ and $B$ be events. The symmetric difference $A \Delta B$ is defined to be the set of all
elements that are in $A$  or $B$ but not both. In logic and engineering, this event is also
called the XOR (exclusive or) of $A$ and $B$. Show that

$$P(A \Delta B) = P(A) + P(B) - 2P(A \cap B)$$


\end{exercise}


\begin{proof}
Using the set difference operator from the previous exercise, we can write $A$ and $B$ as the disjoint unions $A = (A \cap B) \cup (A - B)$ and $B = (B \cap A) \cup (B-A)$. By its definition, $A \Delta B = (A-B) \cup (B-A)$, which are two disjoint sets.

Applying the second axiom of probability, we have that $P(A) = P(A \cap B) + P(A - B)$ and $P(B) = P(A \cap B) + P(B - A)$. Adding the two, we obtain $P(A) + P(B) = 2P(A \cap B) + P(A - B) + P(B - A) = 2P(A \cap B) + P(A \Delta B)$. Rearranging, we obtain the relation from the exercise statement, $P(A \Delta B) = P(A) + P(B) - 2P(A \cap B)$.
\end{proof}

