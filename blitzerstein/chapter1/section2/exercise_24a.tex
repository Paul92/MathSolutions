\begin{exercise}{24a}
A survey is being conducted in a city with 1 million residents. It would be far too
expensive to survey all of the residents, so a random sample of size 1000 is chosen
(in practice, there are many challenges with sampling, such as obtaining a complete
list of everyone in the city, and dealing with people who refuse to participate). The
survey is conducted by choosing people one at a time, with replacement and with equal
probabilities.

Explain how sampling with vs. without replacement here relates to the birthday
problem.
\end{exercise}

\begin{proof}
    The birthday problem is an example of sampling with replacement: we assume that every person can be born independently and with equal probability in each of the 365 days of the year, and so the pool of people considered in the birthday problem is a sample with replacement from the set of the days of a year.

    The presented solution of the birthday problem on page 11 of the textbook solves it by finding the probability of the complement, i.e. the probability that there is no repeated birthday. This is an example of sampling without replacement.
\end{proof}

