\begin{exercise}{61}
There are 100 passengers lined up to board an airplane with 100 seats (with each seat assigned to one of the passengers). The first passenger in line crazily decides to sit in a randomly chosen seat (with all seats equally likely). Each subsequent passenger takes their assigned seat if available, and otherwise sits in a random available seat. What is the probability that the last passenger in line gets to sit in their assigned seat?
\end{exercise}

\begin{proof}
    Without loss of generality, we can assume that the passengers board in the order of the seats, i.e. the $i$'th passenger to board is assigned the $i$'th seat.
    
\vspace{1em}

    We start by making a simplifying observation, that the passengers which find their seat empty and can use it are not relevant for the problem. In order to see why this is true, note that after the first passenger has made his choice, to sit in the randomly seat $i$, the passenger assigned to this sit must make a random choice out of the available seats. They can either choose the seat originally assigned to the first passenger (and the rest of passengers will sit on their assigned seats), or randomly choose another seat $j$. The passengers who sit on their assigned seats are again irrelevant, until is time to seat the passenger assigned to seat $j$.
    
\vspace{1em}

    We notice that a chain forms, of passengers which take each other's seats. At every random choice, the chain ends when a passenger chooses the seat of the first passenger. If the chain ends before the last passenger's turn, then he will sit on his assigned seat. If not, he will sit on the the last remaining seat, which will be the seat of the first passenger. This means that the probability of the last passenger to sit on his own seat is equal to the probability of a previous passenger to have chosen the seat of the first one.
    
\vspace{1em}

    What the above means is that given a subset of passengers which contains the first one, they can be arranged in the increasing order (boarding order) and can make a random choice of the remaining seats such that they seat themselves on the seat of the next passenger in the subset. Implicitly, the passengers of this subset will be the ones which will not sit at their designated places, while the other passengers will sit at their designated places. Therefore, the subset must contain the first passenger and must not contain the last passenger for the last passenger to sit on his designated seat. This implies that there are $2^{98}$ ways of boarding such that the last passenger sits on the designated seat, out of a total of $2^{99}$ boardings, giving a probability of $p = \frac{2^{98}}{2^{99}} = \frac{1}{2}$
\end{proof}

\newpage

