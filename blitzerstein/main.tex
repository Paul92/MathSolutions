\documentclass[a4paper, 11pt]{book}
\usepackage{comment} % enables the use of multi-line comments (\ifx \fi) 
\usepackage{lipsum} %This package just generates Lorem Ipsum filler text. 
\usepackage{fullpage} % changes the margin
\usepackage[a4paper, total={7in, 10in}]{geometry}

%\usepackage{tgadventor} % The font for the entire document can be changed here
%\usepackage{courier}
%\usepackage{charter}
\usepackage{tgcursor}

\usepackage{mathrsfs} % This allows for the use of \mathscr{P} which is the same font used for the powerset notation in Velleman's book.

\newtheorem{corollary}{Corollary}
\usepackage{graphicx}
\usepackage{tikz}
\usetikzlibrary{arrows}
\usepackage{verbatim}
\usepackage[numbered]{mcode}
\usepackage{float}
\usepackage{tikz}
    \usetikzlibrary{shapes,arrows}
    \usetikzlibrary{arrows,calc,positioning}

    \tikzset{
        block/.style = {draw, rectangle,
            minimum height=1cm,
            minimum width=1.5cm},
        input/.style = {coordinate,node distance=1cm},
        output/.style = {coordinate,node distance=4cm},
        arrow/.style={draw, -latex,node distance=2cm},
        pinstyle/.style = {pin edge={latex-, black,node distance=2cm}},
        sum/.style = {draw, circle, node distance=1cm},
    }
\usepackage{xcolor}
\usepackage{mdframed}
\usepackage[shortlabels]{enumitem}
\usepackage{indentfirst}
\usepackage{hyperref}
    
\renewcommand{\thesubsection}{\thesection.\alph{subsection}}

\newenvironment{exercise}[2][Exercise]
    { \begin{mdframed}[backgroundcolor=gray!20] \textbf{#1 #2} \\}
    {  \end{mdframed}}

% Define solution environment

%%%%%%%%%%%%%%%%%%%%%%%%%%%%%%%%%%%%%%%%%%%%%%%%%%%%%%%%%%%%%%%%%%%%%%%%%%%%%%%%%%%%%%%%%%%%%%%%%%%%%%%%%%%%%%%%%%%%%%%%%%%%%%%%%%%%%%%% Original packages, custom environments, and custom commands below
  \usepackage{amsmath,amsfonts,amsthm, amssymb}
  \usepackage{fullpage}
  \usepackage{array}
  \usepackage[all,textures]{xy}
  \usepackage{graphicx}
  \usepackage{alltt}
  \usepackage{listings}
  \usepackage{float}
  \usepackage{tabu}
  \usepackage{longtable}
  \usepackage{lipsum}
  
  \theoremstyle{plain}
  \newtheorem{innercustomgeneric}{\customgenericname}
\providecommand{\customgenericname}{}
\newcommand{\newcustomtheorem}[2]{%
  \newenvironment{#1}[1]
  {%
   \renewcommand\customgenericname{#2}%
   \renewcommand\theinnercustomgeneric{##1}%
   \innercustomgeneric
  }
  {\endinnercustomgeneric}
}

\newcustomtheorem{theorem}{Theorem}
\newcustomtheorem{definition}{Definition}
\newcustomtheorem{lemma}{Lemma}


%  \setlength{\parindent}{0pt}
 
 \newcommand{\pow}{\mathscr{P}}
 \newcommand{\ffam}{\mathcal{F}}
 \newcommand{\family}{\mathcal{F}}
 \newcommand{\gfam}{\mathcal{G}}
 \newcommand{\Ran}{\text{Ran}}
 \newcommand{\Dom}{\text{Dom}}
\newcommand{\solution}{\textit{Solution.}}
 \newcommand{\real}{\mathbb{R}}
 \newcommand{\integer}{\mathbb{Z}}
 \newcommand{\rational}{\mathbb{Q}}
 \newcommand{\nat}{\mathbb{N}}
\begin{document}
\begin{titlepage}
\newcommand{\HRule}{\rule{\linewidth}{0.5mm}}
\center
\HRule \\[0.4cm]
{ \huge \bfseries Introduction to Probability by Blitzstein :\\ Exercise Solutions\\[0.15cm] }
\HRule \\[1.5cm]
\end{titlepage}

\begingroup
\let\cleardoublepage\clearpage
\tableofcontents
\endgroup

\section*{Chapter 1. Sets}

---
layout: page
title:  Introduction to Set Theory
subtitle: Chapter 3
categories: jekyll update
published: true
usemathjax: true
---




---
layout: page
title:  "Introduction to Set Theory"
subtitle: Chapter 6
categories: jekyll update
published: true
usemathjax: true
---




\subsection*{3.3} If $A \neq \emptyset$ is finite and $B$ is countable, then $A \times B$ is countable.

\begin{proof}
We use the same $f,g$ functions whose existence has been shown in the previous exercise, and construct $h: A \times B \rightarrow \mathbb{N} \times \mathbb{N}$ with $h(a,b) = (f(a), g(b))$. This function is injective and hence $|A \times B| \leq |\mathbb{N} \times \mathbb{N}|$, so $A \times B$ is at most countable. Since there is an injection $i: \mathbb{N} \rightarrow A \times B$, $i_a(n) = (a, f^{-1}(n))$ for some $a \in A$, we have that $|\mathbb{N}| \leq |A \times B|$. It follows that $|A \times B| = |\mathbb{N}|$ and the result follows.
\end{proof}


\subsection*{3.4} Derive a "parametric" version of the Recursion Theorem (Theorem 3.6) from the Recursion Theorem.

\begin{proof}

The parametric version of the recursion theorem is defined as:

\begin{itemize}
    \item $f(p, 0) = a(p)$ for all $p \in P$
    \item $f(p, n+1) = g(p, f(p,n), n)$ for all $n \in N$ and $p \in P$
\end{itemize}

Define now a function $F: N \rightarrow A$ with

\begin{itemize}
    \item $F(0) = f(p, 0) = a \in A$
    \item $F(n+1) = G(F(n), n) = g(p, F(n), n)$
\end{itemize}

, which is equivalent to the recursion theorem.

\end{proof}


---
layout: page
title:  "Introduction to Set Theory"
subtitle: Chapter 1
categories: jekyll update
published: true
usemathjax: true
---




---
layout: page
title:  "Introduction to Set Theory"
subtitle: Chapter 2
categories: jekyll update
published: true
usemathjax: true
---




---
layout: page
title:  Introduction to Set Theory
subtitle: Chapter 1
categories: jekyll update
published: true
usemathjax: true
---




---
layout: page
title:  "Introduction to Set Theory"
subtitle: Chapter 1
categories: jekyll update
published: true
usemathjax: true
---







\end{document}
