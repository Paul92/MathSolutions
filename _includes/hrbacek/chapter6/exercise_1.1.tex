\subsection*{1.1}


    Give an example of a linearly ordered set $(L, <)$ and an initial segment $S$ of $L$ which is not of the form $\{x | x <a\}$, for any $a \in L$.

    \begin{proof}
        Let $L = R$ and $S = (-\infty, 0]$. It holds that for any $x \in S$, we have that $a \in S$ for all $a < x$. But also there is no element $e \in R$ such that $S$ could be defined all all reals smaller than $e$. In order to see this, assume to the contrary that such an element $e$ exists. Note that it is necessary that $e>0$, by the construction of $S$. Indeed, for all $x \in S$ we have that $x < e$. But there is $e / 2$, which is smaller than $e$ (since $e$ is positive) but which does not belong to the set $S$. Hence, $S$ could not be described as $\{x | x < e\}$ for any $e$ real.
    \end{proof}



