\subsection*{2.1} A set $X$ is transitive if and only if $X \subseteq P(X)$.

\begin{proof}
    Assume that $X$ is transitive. Then, for every $a \in X$ we have $a \subseteq X$. It follows that $a \in P(X)$. Since we have for all $a \in X$, $a \in P(X)$, it follows that $X \subseteq P(X)$.

    Conversely, assume that $X \subseteq P(X)$. Pick some $a \in X$. Since $X \subseteq P(X)$, $a \subseteq X$. Hence, every element of $X$ is also a subset of $X$, and hence $X$ is transitive.
\end{proof}

