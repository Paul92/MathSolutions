\subsection*{2.7} If a set of ordinals $X$ does not have a greatest element, then $\sup X$ is a limit ordinal.

\begin{proof}
    Let $X$ be a set of ordinals without a greatest element. This implies that $\sup X = \bigcup X \notin X$. Assume to the contrary that $sup X$ is a successor ordinal, that is, there is some $\alpha$ such that $S(\alpha) = \sup X$. But then it is necessary that $\alpha \in X$, since $\alpha \in \sup X$. By construction, there is no other ordinal between $\alpha$ and $\sup X$. Since $\alpha \in X$, $\alpha$ is therefore a greatest element, which is a contradiction and the conclusion follows.
\end{proof}

