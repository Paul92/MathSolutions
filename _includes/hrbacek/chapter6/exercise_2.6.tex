\subsection*{2.6} An ordinal $a$ is a natural number if and only if every nonempty subset of $a$ has a greatest element.

\begin{proof}
    Every natural number is a finite ordinal. Hence, every nonempty subset of $a \in N$ is a finite set. But the ordinals are well ordered, and hence all elements are comparable. In a well order, finite sets have greatest and lowest element, and hence the conclusion follows.

    Conversely, assume that every nonempty subset of $a$ has a greatest element. The smallest ordinals are the natural numbers, where this property clearly holds. The first transfinite ordinal is $N$ itself. $N$ does not have a greatest element, so the hypothesis does not hold. The question is whether it holds for any other transfinite number. But any transfinite number is constructed by appending elements to $N$. In other words, $N$ an initial segment for any transfinite ordinal. It follows that all transfinite ordinals have a subset without a greatest element. Hence, the only ordinals for which the hypothesis holds are the naturals.
\end{proof}

