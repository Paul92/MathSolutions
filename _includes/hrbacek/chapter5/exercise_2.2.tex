\subsection*{2.2} A real number $x$ is algebraic if it is a solution of some polynomial equation with integer coefficients. If $x$ is not algebraic, it is called transcendental. Show that the set of all algebraic numbers is countable and hence
the set of all transcendental numbers has cardinality $2^{\aleph_0}$.

\begin{proof}
Each polynomial is uniquely undefined by an n-tuple of natural numbers $(a_1, \dots, a_n)$. We can uniquely identify this tuple with the natural number $\sum_{i=1}^n a_ip_i$, where $p_i$ is the $i$-th prime number. Hen
ce, the set of polynomials with natural numbers is countable, with a cardinality of $\aleph_0$. Each polynomial has a finite number of solutions at most equal to its degree, and hence there are at most $\aleph_0$ solutions for each polynomial. Hence, there are at most $\aleph_0 \cdot \aleph_0 = \aleph_0$ algebraic numbers, and the conclusion follows.
\end{proof}

