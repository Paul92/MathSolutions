\begin{exercise}{10a}
    To fulfill the requirements for a certain degree, a student can choose to take any 7 out
of a list of 20 courses, with the constraint that at least 1 of the 7 courses must be a
statistics course. Suppose that 5 of the 20 courses are statistics courses.

\vspace{1em}
How many choices are there for which 7 courses to take?
\end{exercise}

\begin{proof}
    The total number of course choices, disregarding the requirement for a statistic class, are $20 \choose 7$ = 77520. From these, we should remove the ones that do not contain any statistics class. These are basically the course choices from the non-statistic classes, with
    $15 \choose 7$ = 6435. This means that there are 77520 - 6435 = 71085 course choices that validate the requirement of one statistic course.
\end{proof}

