\begin{exercise}{44}
Let $A$ and $B$ be events. The difference $B - A$ is defined to be the set of all elements of $B$ that are not in $A$. Show that if $A \subseteq B$, then

$$P(B-A) = P(B) - P(A)$$

directly using the axioms of probability.
\end{exercise}

\begin{proof}
    Note that $B-A$ and $A$ are, by the definition of the set difference, disjoint sets. This means that we can apply the second axiom of the probability to obtain $P((B-A) \cup A) = P(B-A) + P(A)$. But also, $(B-A) \cup A - B$ and hence we have that $P(B) = P(B-A) + P(A)$. Rearranging, we obtain the relation from the exercise statement, $P(B-A) = P(B) - P(A)$.
\end{proof}

