\begin{exercise}{22}
    The Dutch mathematician R.J. Stroeker remarked:
    \begin{quote}
        Every beginning student of number theory surely must have marveled at the miraculous fact that for each natural number $n$ the sum of the first $n$ positive consecutive cubes is a perfect square.
    \end{quote}
    Furthermore, it is the square of the sum of the first $n$ positive integers! That is,
    $$1^3 + 2^3 + ... + n^3 = (1+2+...+n)^2.$$
    Usually this identity is proven by induction, but that does not give much insight into why the result is true, nor does it help much if we wanted to compute the left-hand side but didn't already know this result. In this problem, you will give a story proof of the identity.
    \begin{enumerate}
        \item Give a story proof of the identity
        $$1+2+...+n=\binom{n+1}{2}.$$
        \item Give a story proof of the identity
        $$1^3+2^3+...+n^3 = 6\binom{n+1}{4} + 6\binom{n+1}{3} + \binom{n+1}{2}.$$
    \end{enumerate}
    It is then just basic algebra (not required for this problem) to check that the square of the right-hand side in (a) is the right-hand side in (b).
\end{exercise}
\begin{proof}
    \begin{enumerate}
        \item Consider a round-robin tournament. To find the number of games played, we can just take $\binom{n+1}{2}$, the total number of pairs in $n+1$ contestants. But we can also reason as follows: player $1$ has to play $n$ people, so those are $n$ games. Then player $2$ has to play $n-1$ people (since he already played player $1$), so those are $n-1$ additional game. Then player $3$ has to play $n-2$ people and so on, resulting in $1+2+3+...+n$ games.
        \item Imagine choosing a number between $1$ and $n$ and then choosing $3$ numbers between $0$ and $n$ smaller than the original numbers, this can happen in $n^3$ ways, so we get in total $1^3 + 2^3 + ... + n^3$ possibilities. On the other hand, there are the following cases:
        \begin{enumerate}
            \item All $4$ chosen numbers are distinct, then we just choose $4$ numbers out of $n+1$, we still have to order the $3$ smaller numbers based on when they were chosen, so we get in total $3!\binom{n+1}{4}$.
            \item Two number coincide, that means we choose $3$ numbers out of $n+1$, and then we still have to order the $3$ smaller numbers based on when they were chosen, so we get $3!\binom{n+1}{3}$.
            \item All three smaller numbers coincide, in this case we just choose $2$ numbers out of $n+1$ to get $\binom{n+1}{2}$.
        \end{enumerate}
        Taking the three cases together we get the required answer.
    \end{enumerate}
\end{proof}


