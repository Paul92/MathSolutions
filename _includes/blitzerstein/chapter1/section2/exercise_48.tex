\begin{exercise}{48}
Arby has a belief system assigning a number $P_{Arby}(A)$ between 0 and 1 to every event $A$ (for some sample space). This represents Arby's degree of belief about how likely $A$ is to occur. For any event $A$, Arby is willing to pay a price of $1000 \cdot P_{Arby}(A)$ dollars to buy a certificate such as the one shown below.

\vspace{1em}

\noindent\fbox{
    \parbox{\textwidth}{
    \textbf{Certificate}
        The owner of this certificate can redeem it for \$1000 if $A$ occurs. No value if $A$ does not occur, except as required by federal, state or local law. No expiration date.
    }
}
\vspace{1em}

Likewise, Arby is willing to sell such a certificate at the same price. Indeed, Arby is willing to buy or sell any number of certificates at this price, as Arby considers it the "fair" price.

Arby stubbornly refuses to accept the axioms of probability. In particular, suppose that there are two disjoint events $A$ and $B$ with

$$P_{Arby}(A \cup B) \neq P_{Arby}(A) + P_{Arby}(B)$$

Show how to make Arby go bankrupt, by giving a list of transactions Arby is willing to make that will guarantee that Arby will lose money (you can assume it will be known whether $A$ occurred and whether $B$ occurred the day after any certificates are bought/sold).
\end{exercise}


\begin{proof}
    Let's assume there are three certificates, $A$, $B$ and $C$, which correspond to the events $A$, $B$ and $B$, respectively, with $A$ and $B$ disjoint. Also, assume that $C = A \cup B$.

    If Arby believes that 
    $$P_{Arby}(A \cup B) \neq P_{Arby}(A) + P_{Arby}(B)$$

    then we might have that

    $$P_{Arby}(C) = P_{Arby}(A \cup B) < P_{Arby}(A) + P_{Arby}(B)$$

    This means that Arby will think that $C$-certificates are worth less than a pair of $A$ and $B$-certificates, so he will happily make this exchange, adding some money from his pocket. This will be money Arby loses, since the value of a $C$ certificat is, in fact, exactly equal to the value of the pair of $A$ and $B$ certificates Arby buys.

    \vspace{1em}

    Consider the alternative case, where 
    
     $$P_{Arby}(C) = P_{Arby}(A \cup B) > P_{Arby}(A) + P_{Arby}(B:)$$

    This is the opposite of the previous case. Here, Arby thinks that a pair of $A$ and $B$-certificates are worth less than a $C$-certificate, and hence he is willing to sell the pair, add some more money and buy a $C$-certificate. This is again, money he will loose, since the value of the $C$ certificate he buys is exactly the value of the pair of $A$ and $B$ certificates he buys.
\end{proof}


